\documentclass{article}

\usepackage[version=3]{mhchem} % Package for chemical equation typesetting
\usepackage{siunitx} % Provides the \SI{}{} and \si{} command for typesetting SI units
\usepackage{graphicx} % Required for the inclusion of images
\usepackage{natbib} % Required to change bibliography style to APA
\usepackage{amsmath} % Required for some math elements 
\usepackage[utf8x]{inputenc}
\usepackage{hyperref}

\renewcommand{\labelenumi}{\alph{enumi}.} % Make numbering in the enumerate environment by letter rather than number (e.g. section 6)

%\usepackage{times} % Uncomment to use the Times New Roman font

%----------------------------------------------------------------------------------------
%	DOCUMENT INFORMATION
%----------------------------------------------------------------------------------------

\title{AM4L: Applications Mobiles \\ Réalisation d'une application Android (ECAlendar) \\ ECAM Bruxelles} % Title

\author{Charles \textsc{Vandevoorde} (13019)\\
		Lorenzo \textsc{Riga} (13018)\\
		Antoine \textsc{Vander Meiren} (12088)\\
		Gaetano \textsc{Giordano} (12054)\\
		Sylvain \textsc{Alonso} (12150)\\} % Author name

\date{\today} % Date for the report

\begin{document}
	
	\maketitle % Insert the title, author and date
	
	
	\section*{Introduction}
	\hspace{0.45cm}
	Dans le cadre du cours d'applications mobiles, nous avons réalisé une application Android "ECAlendar" à l'aide d'Android Studio. Cette dernière utilise l'API de 'Ecam Calendar" afin de permettre aux étudiants de consulter leurs horaires directement sur leurs GSM.\\
	
	Dans ce rapport, nous allons présenter la manière dont nous nous sommes organisés en groupe mais aussi la structure de notre application ainsi qu'un listing de ses fonctionnalités. 
	
	\section{Organisation du groupe}
	\hspace{0.45cm}
	 Nous avons hébergé notre projet sur Github de manière à faciliter le travail en groupe et nous sommes répartis les tâches comme suit lors de la première séance:
	 \begin{itemize}
	 	\item Charles:
	 	\item Lorenzo:
	 	\item Antoine:
	 	\item Gaetano:
	 	\item Sylvain: intégration de la base de données SQLite et gestion des clics (affichage des détails lorsque l'utilisateur appuie sur un élément).
	 \end{itemize}
	 
	
	\section{Présentation de l'architecture (diagramme de classes)}
	\hspace{0.45cm}
	
	\section{Listing des fonctionnalités}
	\hspace{0.45cm}
	
	\section{Conclusion}
	\hspace{0.45cm}

	
	
	
	
	
\end{document}